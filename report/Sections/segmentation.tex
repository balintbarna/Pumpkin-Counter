\documentclass[../Head/Main.tex]{subfiles}
\begin{document}
\section{Segmentation of Images}
exercises 2 and 4

Using the color information of the pixels, we can estimate if they are pumpkin or not.
The simplest way to do that is to check if the color values fall inside a predetermined range.

To get usable results, we need to choose the limits wisely.
We did that by looking at the histograms of the color channels in the original image.
In the ranges we used the mean +- one standard deviation.

Slight natural variations in color can cause big differences in the performance of our detection.
In order to get a more consistent look on the pumpkins we used a blurring algorithm.
Our choice of algorithm is median blur. It performs nicely in terms of smoothing while also keeping hard contours.

Here is a showcase of the filter in action. Original (left) and filtered (right). The filter kernel size is set to 9 pixels.

.. image:: images/ori_smol.jpg
    :width: 49 %
.. image:: images/blurred_smol.jpg
    :width: 49 %


\end{document}